\subsection{Description}
The flange effect is very popular among musicians, as it adds interest to a track in a periodic way. Flanging sounds as if an original instrument's sound is changed from a natural to artificial and back, or as if the audio has been combined with a jet sound.

\subsection{Applications}
The flange effect is often used to add a periodic, synthetic sound to a track. This adds an interesting sound variation to most any musical track.

\subsection{Principles of Operation}
The flanger is quite simple in concept: it consists of a variable delay which is controlled by a low frequency oscillator (LFO). The delay output is then summed with a non-delayed version and output to the destination.

\subsection{Implementation Notes}
To implement this effect, we used MATLAB (see \href{run:../Flange.m}{Flange.m}). Starting with the basic audio file implementation from class to handle importing and playing the audio files. We built upon this and added both a new line in the step function for adding the two signals to be produce together with varying gains. A block diagram can be seen in Figure \ref{fig:flange-block-diagram}.
\begin{figure}[ht]
	\centering
	\includegraphics[scale=0.7]{flange.png}
	\caption{Flange Block Diagram}
	\label{fig:flange-block-diagram}
\end{figure}

\subsection{Demo and Discussion}
Flanging can be used in a variety of ways to produce a wide range of sound effects from the same track.We adjusted the maximum delay using a
\href{run:../OutputAudio/flange_22-004 Original Guitar_{freq=0.4Hz}{delay_max=4ms}.wav}{4ms}
delay and an
\href{run:../OutputAudio/flange_22-004 Original Guitar_{freq=0.4Hz}{delay_max=8ms}.wav}{8ms}
 delay using a 0.4Hz LFO. We then tried used a 1Hz LFO with a
\href{run:../OutputAudio/flange_22-004 Original Guitar_{freq=1Hz}{delay_max=4ms}.wav}{4ms}
and an
\href{run:../OutputAudio/flange_22-004 Original Guitar_{freq=1Hz}{delay_max=8ms}.wav}{8ms}
delay.
\\ \\
The flanging sound is very noticeable, but not extremely deep. Many flangers have feedback in their delay line, creating resonances in the track which can increase the effect's presence in the track.

We then implemented a stereo flanger, using a second LFO with a $\pi /2$ phase shift. The results of our adjustments include maximum delay using a
\href{run:../OutputAudio/stereoflange_22-004 Original Guitar_{freq=0.5Hz}{delay_max=4ms}.wav}{4ms}
delay and an
\href{run:../OutputAudio/stereoflange_22-004 Original Guitar_{freq=0.5Hz}{delay_max=8ms}.wav}{8ms}
 delay using a 0.4Hz LFO. We then tried used a 1Hz LFO with a
\href{run:../OutputAudio/stereoflange_22-004 Original Guitar_{freq=1Hz}{delay_max=4ms}.wav}{4ms}
and an
\href{run:../OutputAudio/stereoflange_22-004 Original Guitar_{freq=1Hz}{delay_max=8ms}.wav}{8ms}
delay.

\subsection{Further Exploration}
Something to investigate is how this effect would change if it also had a chorus aspects, placing the flange effect at different rates on different chorus channels. This would be an interesting demonstration of two effects from this chapter. A cool way to further investigate this effect is to watch this neat YouTube video about the flange effect \href{https://www.youtube.com/watch?v=Ici_YOVDl_0}{here}.
