\subsection{Description}
The flange effect is verypopular among musicians, as it adds interest to a track in a perioduic way. Flanging sounds as if an original instrument's sound is changed from a natural to artificial and back 

\subsection{Applications}
The flange effect is often used to add a periodic, synthetic sound to a track. This adds an interesting sound variation to most any musical track.

\subsection{Principles of Operation}
The flanger is quite simple in concept: it consists of a variable delay which is controlled by a low frequency oscillator (LFO). The delay output is then summed with a non-delayed version and output to the destination.

\subsection{Implementation Notes}
To implement this effect, we used MATLAB. Starting with the basic audio file implementation from class to handle importing and playing the audio files. We built upon this and added both a new line in the step function for adding the two signals to be produce together with varying gains. 

\subsection{Demo and Discussion}
Flanging can be used in a varitey of ways to produce a wide range of sound effects from the same track.


The flanging sound is very noticeble, but not extremely deep. Many flangers have feedback in their delay line, creating resonances in the track which can increase the effect's noticeability in the track. 


\subsection{Further Exploration}
Somethign to investigate is how this effect would change if it also had a chorus aspects, placing the flange effect at different rates on different chorus channels. This would be an interesting deomnstration of two effects from this chapter.
