\subsection{Description}
This mono effect mimics tremolo or vibrato created by musicians on instruments like violins or guitars. Tremolo is create by exploiting the same technique for amplitude modulation (AM) with a low frequency carrier signal.

\subsection{Applications}
Tremolo can be used in mixing applications to create a natural vibrato to a flat sounding track. However, its use is extremely apparent in electric guitar tracks that use it with a higher rate.

\subsection{Principles of Operation}
The operation is fairly similar to amplitude modulation. However, while AM signals might have carrier frequencies in MHz, tremolo carrier signals only range between 1 and 10 Hz, typically. The carrier signal is defined in Equation \ref{eq:tremolo-carrier}. A block diagram on tremolo can be found in Figure \ref{fig:tremolo-block-diagram}.
\begin{equation}
    m[n] = 1 + \alpha \cos(\omega n)
    \label{eq:tremolo-carrier}
\end{equation}
\begin{figure}[ht]
    \centering
    \includegraphics[scale=0.5]{tremolo-block-diagram.png}
    \caption{Tremolo Block Diagram}
    \label{fig:tremolo-block-diagram}
\end{figure}

\subsection{Implementation}
Tremolo is implemented in MATLAB using an LFO to generate the $\alpha \cos(\omega n)$ term in Equation \ref{eq:tremolo-carrier}. Time domain processing is then applied to mix the carrier signal with the input signal.

\subsection{Demo and Discussion}
The effect was applied to this \href{run:../InputAudio/22-004 Original Guitar.wav}{original guitar track}. The effect can be heard for a carrier frequency of \href{run:../OutputAudio/tremolo_22-004 Original Guitar_{depth=1}{f_LFO=5Hz}.wav}{5 Hz}. To hear how changing the carrier frequency changes the output, the effect was applied for \href{run:../OutputAudio/tremolo_22-004 Original Guitar_{depth=1}{f_LFO=1Hz}.wav}{1 Hz}, \href{run:../OutputAudio/tremolo_22-004 Original Guitar_{depth=1}{f_LFO=10Hz}.wav}{10 Hz}, and \href{run:../OutputAudio/tremolo_22-004 Original Guitar_{depth=1}{f_LFO=20Hz}.wav}{20 Hz} as well.

\subsection{Further Exploration}
A great use of tremolo can be heard in Link Wray's \href{https://www.youtube.com/watch?v=ucTg6rZJCu4}{Rumble}. Though the effect is barely in use at first, as the song progresses, the guitarist turns up the depth ($\alpha$) on the tremolo until it is extremely obvious.